\documentclass[10pt, letterpaper]{article}
\usepackage[utf8]{inputenc}

\title{CSE305/Project - Fast Fourier Transform}
\author{Aya Hankir, Jiwon Park, Minh Khoa Tran}
\date{June, 2021}

\begin{document}

\maketitle

\section{Overview}
The goal of the proejct is to explore the Fast Fourier Transform (FFT).

Particularly, we manage to explore 2 objectives:
\begin{itemize}
	\item Compression of weather data using FFT
	\item Parallelization of the FFT using thread pools
\end{itemize}


\section{Cooley-Tukey Radix-2 Algorithm}
The FFT version that we implement is the Cooley-Tukey Radix-2 [1].

It has the theoretical complexity is $O(n \log n)$, but it is limited to
only the cases when $n$ is a power of 2, or $n = 2^m$ with some $m$.

We implement 2 versions of the version:
\begin{itemize}
	\item \textbf{Recursion}: This version uses the divide-and-conquer
		method. It is simpler to implement and understand the
		codeflow, but it has some overheads from the recursive
		calls.

		This corresponds to \texttt{fft\_fw\_rec} and
		\texttt{fft\_bw\_rec} functions in \texttt{lib.cc}.

	\item \textbf{In-place}: This version uses the Fast Bit-Reversal
		Algoritm to avoid recursive calls. It is faster (see
		benchmark), but it requires a special algorithm.

		This corresponds to \texttt{fft\_fw\_inp} and
		\texttt{fft\_bw\_inp} functions in \texttt{lib.cc}.

		The Fast Bit-Reversal Algorithm we use is from Cornell
		University [2].
\end{itemize}


\section{Parallel FFT}
The in-place FFT algorithm in theory can be improved using parallization.
The scheme we use as the reference is from Tel Aviv University [3].

This corresponds to \texttt{fft\_fw\_par} and \texttt{fft\_bw\_par}
functions in \texttt{lib.cc}.

In each stage of the main loop, we can launch multiple threads to perform
the butterfly operations on each disjointed segments. We use a thread pool
model in \texttt{pool.cc} to manage these threads.


\section{Data Compression}
An application of the FFT is data compression.

By applying FFT on an array, we can omit the higher-frequency coefficients
to save storage space. The original data can be estimated using the reverse
transformation.

Note that if the array size is not a power of 2 (constraint of the Radix-2
algorithm), we could pad it with some dummy data.

\subsection{Dataset}
We use historical weather data as the dataset for the compression.

The data used is the mean temperature measured in Orly, France from
2000 to 2019 [4]. It contains 7305 entries, which would be padded with 0 to
the size of 8192, in order to meet the size constraint.

\subsection{Efficiency}
The algorithm shows poor result in practice.

Particularly, even we keep 8000 coefficients out of 8192, which is more
than the number of terms in the original data, we still see an average
error of 2.65 C degree, with the maximum error of 7.9 C degree.

An observation we notice in testing is that the coefficients after the
FFT tend to be very large, with their average normed in the order of 8-10.
This significantly impacts the quality of the approximation. Further
testing also shows that the padding data does not correlate with the
quality.

The Discrete Cosine Transform, a related algorithm, might be an interesting
choice instead of the FFT.


\section{Benchmark}
We calculate and compare the running time of all 3 versions of the FFT. We
also vary the number of workers in the worker pool.

The number is the average running time over 100 runs on an Intest i5-4300M
CPU with 4 cores.

- Array Size: 8192
\begin{itemize}
	\item 1-thread Recursive FFT: 4.3 ms
	\item 1-thread In-place FFT (inline): 2.7 ms
	\item 1-thread FFT: 4.5 ms
	\item 2-thread FFT: 4.2 ms
	\item 4-thread FFT: 3.9 ms
	\item 8-thread FFT: 6.1 ms
	\item 16-thread FFT: 10.3 ms
	\item 32-thread FFT: 19.3 ms
\end{itemize}

- Array Size: 65536
\begin{itemize}
	\item 1-thread Recursive FFT: 45.3 ms
	\item 1-thread In-place FFT (inline): 26.6 ms
	\item 1-thread FFT: 35.6 ms
	\item 2-thread FFT: 30.5 ms
	\item 4-thread FFT: 26.8 ms
	\item 8-thread FFT: 27.4 ms
	\item 16-thread FFT: 33.8 ms
	\item 32-thread FFT: 47.2 ms
\end{itemize}

The result shows insignificant improvement for the parallelization. A
possible reason for this is the size of the array is still small, and the
speed up gained is lost from in the overhead of the thread pool. Further
testing is required.


\section{References}
\begin{enumerate}
	\item An algorithm for the machine calculation of complex Fourier
		Series, James W. Cooley and John W. Tukey

		https://doi.org/10.2307/2003354

	\item Fast Bit-Reversal Algorithms, Anne Catherine Elster

		https://folk.idi.ntnu.no/elster/pubs/elster-bit-rev-1989.pdf

	\item A parallel FFT on an MIMD machine, Amir Averbuch, Eran
		Gabber, Boaz Gordissky and Yoav Medan

		https://doi.org/10.1016/0167-8191(90)90031-4

	\item European Climate Assessment \& Dataset.

		https://www.ecad.eu
\end{enumerate}

\end{document}
